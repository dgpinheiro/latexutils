\PassOptionsToPackage{table}{xcolor}

% Definições do tipo de documento
\documentclass[12pt,a4paper,ruledheader,espacoumemeio,floatnumber=continuous]{abnt}


% Pacote que permite utilizar vários arquivos de bibliografia em diferentes seções
\usepackage[defaultbib]{bibtopic}
\bibliographystyle{my-abnt-alf}

% Previne que as figuras saiam fora da seção
\usepackage[section,subsection,subsubsection]{extraplaceins}

\usepackage{alltt}

\usepackage{tikz}
\usetikzlibrary{shapes,shadows,arrows}

\usepackage{enumitem}

\usepackage{xspace}

\let\OldTexttrademark\texttrademark
\renewcommand{\texttrademark}{\OldTexttrademark\xspace}%


% Idioma e acentos
\usepackage[brazil]{babel}
\usepackage[utf8]{inputenc}

% Usa fonte European Computer Modern (EC) font
\usepackage{type1ec}

\renewcommand*\thesection{\arabic{section}}

\usepackage{textcase}
\usepackage[explicit]{titlesec}

\titleformat{\section}
  {\normalfont\bfseries}{\thesection}{1em}{\MakeTextUppercase{#1}}

\titlespacing{\section}{0pt}{*0}{*0}

% novo comando \src para incluir o nome de programas
\newcommand{\src}[1]{%
        {\tt [#1]}
}

% http://www.mat.uc.pt/~pedro/LaTeX/
\def\bo{\mbox{\raise .35em \hbox{\underline{\scriptsize o}\ }}}
\def\ba{\mbox{\raise .35em \hbox{\underline{\scriptsize a}\ }}}
\def\Exmo{Ex\mbox{\raise .35em \hbox{\underline{\footnotesize mo}\ }}}
\def\Exmos{Ex\mbox{\raise .35em \hbox{\underline{\footnotesize mos}\ }}}
\def\Excia{Ex\mbox{\raise .35em \hbox{\underline{\footnotesize cia}\ }}}

\usepackage{amsmath}


% Coloca quebra de linhas em URLs
\usepackage{url}

\usepackage{multirow} % Permite tabelas com multiplas linhas colapsadas.
\usepackage{booktabs} % Permite o uso de tabelas profissionais (\toprule,\bottomrule, etc).

\usepackage[footnotesize,bf,up,center]{caption} % Diminui o tamanho da letra e faz o texto (Figura N ou Tabela N) em negrito
\captionsetup[table]{justification=justified,singlelinecheck=false} % Justifica as legendas (captions) das tabelas.
\captionsetup[figure]{justification=justified,singlelinecheck=false} % Justifica as legendas (captions) das figuras.
\setlength{\belowcaptionskip}{5pt} % Configura o espa?amento entre a legenda e a tabela ou figura

\usepackage{longtable}
\setlongtables % keeps the width uniform across both pages

\usepackage{pdflscape} % Permite rotacionar a p?gina para Paisagem (landscape)
\usepackage{rotating} % Permite rotacionar tabelas (sideways)

\usepackage{supertabular}
\usepackage[paginas=nao,ordem=oc]{tabela-simbolos} % Usado para gerar a Lista de Siglas
\newindex{default}{idx}{ind}{Índice Remissivo} % Gera um segundo índice

\usepackage[final]{pdfpages}

\usepackage{siunitx}

\usepackage{colortbl}

\usepackage[table]{xcolor}

\usepackage{fixltx2e}
\usepackage[flushleft]{threeparttable}

\usepackage{acronym}

\usepackage{tabularx}
\usepackage[margin=2cm]{geometry}

\usepackage[T1]{fontenc}

\hyphenation{circ-RNAs}
\hyphenation{va-li-da-ção}
\hyphenation{se-quen-ci-a-do}

\newcommand{\TM}{\textsuperscript{\tiny\texttrademark}}
\newcommand{\RM}{\textsuperscript{\tiny\textregistered}}

\definecolor{lightgray}{gray}{0.9}

%\usepackage{hyperref}

\begin{document}

	\DeclareGraphicsExtensions{.jpg,.png,.gif}

	\begin{titlepage}
\begin{figure}[ht]
\centering
\end{figure}
	\begin{center}
		\Large
		\textbf{Projeto de Pesquisa (PIBIC)}
		\\[1cm]
		Faculdade de Ciências Agrárias e Veterinárias\\
		Universidade Estadual Paulista ``Júlio de Mesquita Filho''
		\\[3cm]
		% Título 
		{\bfseries {\LARGE 
		Título do Projeto
		} }\\[3cm]
	
	\end{center}

\vspace{2cm}
  \begin{center}%
      {\LARGE \sc Nome do aluno}\\[2cm]
  \end{center}

  \begin{flushright}
    \begin{minipage}[h]{12.5cm}

\begin{tabular}{ l l }
  Orientador & Prof. Dr. Daniel Guariz Pinheiro\\
   &  \\
   &  Departamento de Tecnologia\\
   &  Faculdade de Ciências Agrárias e Veterinárias (FCAV)\\
   &  Universidade Estadual Paulista ``Júlio de Mesquita Filho'' (UNESP)\\
\end{tabular}

    \end{minipage}
  \end{flushright}

\vspace{2cm}
  \begin{center}%
	      Jaboticabal/SP - Mês/Ano
  \end{center}

\end{titlepage}

	
	{\let\clearpage\relax\section{Introdução}}\label{sec:introducao}
    	Introdução \cite{pmid20041221}

	{\let\clearpage\relax\section{Objetivos}}\label{sec:objetivos}
    	\input{section/objetivos}
    
	{\let\clearpage\relax\section{Metodologia}}\label{sec:metodologia}
	\input{section/metodologia}
	
	{\let\clearpage\relax\section{Cronograma de Execução}}\label{sec:cronograma}
	As atividades propostas neste projeto serão realizadas em um prazo previsto de
12 meses, conforme a Tabela~\ref{tab:cronograma}.

\begin{table}[h!]
	\center
	\begin{tabular}[ht!]{| c | c | c | c | c | c | c | c | c | c | c | c | c |} \hline
    	\multirow{2}{*}{\textbf{Atividade}} & \multicolumn{12}{|c|}{\textbf{Meses (2014/2015)}} \\
        \cline{2-13}
     	& \textbf{1\bo} & \textbf{2\bo} & \textbf{3\bo} & \textbf{4\bo} & \textbf{5\bo} & \textbf{6\bo} &  \textbf{7\bo} & \textbf{8\bo} & \textbf{9\bo} & \textbf{10\bo} & \textbf{11\bo} & \textbf{12\bo} \\
        \hline
	
	\end{tabular}
        \caption{Cronograma das atividades}
	\label{tab:cronograma}
\end{table}

	
	{\let\clearpage\relax\section{Plano de Atividades}}\label{sec:atividades}
	\input{section/atividades}
	
 	{\let\clearpage\relax\section{Bibliografia}}\label{sec:bibliografia}
    
	\begin{btSect}{bibliografia.bib}
        	\btPrintCited
	\end{btSect}

\end{document}
